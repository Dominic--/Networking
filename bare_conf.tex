\documentclass[conference]{IEEEtran}

\ifCLASSINFOpdf
  % \usepackage[pdftex]{graphicx}
  % declare the path(s) where your graphic files are
  % \graphicspath{{../pdf/}{../jpeg/}}
  % and their extensions so you won't have to specify these with
  % every instance of \includegraphics
  % \DeclareGraphicsExtensions{.pdf,.jpeg,.png}
\else
  % or other class option (dvipsone, dvipdf, if not using dvips). graphicx
  % will default to the driver specified in the system graphics.cfg if no
  % driver is specified.
  % \usepackage[dvips]{graphicx}
  % declare the path(s) where your graphic files are
  % \graphicspath{{../eps/}}
  % and their extensions so you won't have to specify these with
  % every instance of \includegraphics
  % \DeclareGraphicsExtensions{.eps}
\fi

% correct bad hyphenation here
\hyphenation{op-tical net-works semi-conduc-tor}


\begin{document}

\title{Draft for Networking Paper}


\author{\IEEEauthorblockN{Michael Shell}
\IEEEauthorblockA{Tsinghua University}
\and
\IEEEauthorblockN{Homer Simpson}
\IEEEauthorblockA{Tsinghua University}
\and
\IEEEauthorblockN{James Kirk}
\IEEEauthorblockA{Tsinghua University}}


% make the title area
\maketitle

% As a general rule, do not put math, special symbols or citations
% in the abstract
\begin{abstract}
robust, performance, energy
\end{abstract}

\IEEEpeerreviewmaketitle

\section{Introduction}
We aim to find the ``robust'' route for all possible traffic matrix, not only consider performance but also energy 
consumption in network.

\section{Motivation}
Things seem to be not so obviously simple. Suppose two hosts named host A and host B, and there are three links between 
them, respectively, capacity with 2M, 3M and 5M. Now traffic matrix comes, with 1M from A to B. we regard the capacity 
as the power of the link, and the minimum maximum utilization of network links as a metric of the network performance. 
There are two directions for operating this example. One consider the min power consumption except for the 
utilization, it is obviously that we should close the larger power consuming links, such as the 5M and 3M links, and 
all the traffic go through the 2M link. In this way, the min max utilization of network is 0.5 and the power consuming 
is 2 units (means the power difference come from the link mainly). The other consider the power except for the 
utilization reversely, so we should split the 1M traffic to three parts, 0.2M across 2M link, 0.3M across 3M link and 
0.5M across 5M, consequently with a min max utilization of 0.1, but the power is 10 units however.

Two directions mentioned above both are extremely single-consideration. Previous researchers solve the problem more 
considerable, include ``GreenTE'' and ``a\%-green is engouh''. The former set a threshold of min max utilization, 
close links as many as possible to achive the most power saving. And the latter one set a destination of the power 
saving, calculate optimal route for get the min max utilization. Two work have their restriction, which both need a 
specific traffic matrix that as base of their optimization.

But the need of precise current traffic matrix shound be carefully checked. Although some researcher contribute to this 
area, the real precise traffic matrix still be a challenge. Take a step back, the dynamic of traffic matrix is more 
diffcult even if we obtain the precise current one. Futhermore, ISP will not want to change their route policy 
frequently, as it will result in other route failure possiblly. So our question is that : Is there exist a route both 
satisfy power and utilization requirement for any traffic matrix given?

David Applegate propose a method for obtain a route wihich is ``robust'' to variations in demands for a specific network topology. 

\section{Model}
We model the network as a undirected graph $G = (V, E)$, where $V$ is the set of vertices (i.e., routers and end hosts), 
and $E$ is the set of links (either link between routers or router and end-host). In graph $G$, two vertices $u$ and $v$
are called connected if $G$ contains a path from $u$ to $v$. Then We say graph $G$ is connected, if and only if 
every pair of vertices in the graph is connected.

Let $\theta(G) = \{ (V, E - \{ e \}) | e \in E \}$ denote the network set after closing/removing the link $e$ from
$G$. Then, choosing the connected graph from $\theta(G)$ to consist a new set, denoted by 
$\Theta(G) = \{G | G \in \theta(G) && G is connected\}$. We call $\Theta(G)$ as successor of $G$.

A $traffic matrix$ (abbreviation as TM below) is the set of traffic of each Origin-Destination(OD) pair in 
network $G$, and a $routing$ specifies how traffic of each OD pair is routed across the network. Usually, there are 
multiple paths for each OD pair and each path routes a fraction of the traffic. Let $m$ denote the $traffic matrix$, 
which can be represented by a set of trinary group like $(a, b, d_{ab}$, where $a$ and $b$ is the origin and 
destination of pair respectively, $d_{ab}$ is the traffic demand of the OD pair. 

Let $r$ denote the $routing$ mentioned above, which is specified by a set of values $f_{ab}(i,j)$ that specifies the 
fraction of demand from $a$ to $b$ that is routed on the link $(i,j)$. So an OD pair contribute to the traffic of 
link $(i,j)$ is $d_{ab}f_{ab}(i,j)$, and all the traffic across link $(i,j)$ can be calculated as :
\begin{equation}
	\sum_{(a,b,d_{ab}\in m)} d_{ab}f_{ab}(i,j)
\end{equation}

Futhermore, we define the utilization of link as traffic acorss the link divide capacity of the link, as ;
\begin{equation}
	u_{ij} = \frac{\sum_{a,b} d_{ab}f_{ab}(i,j)}{cap_{ij}}
\end{equation}
where $cap_{ij}$ si the capacity of the link $(i,j)$.

A common metric for the performance of a given routing with respect to a certain TM is the $maximum link utilization$.
This is the maximum utilization of link over all ones, Formally, the maximum link utilization of a routing $r$ on 
TM $m$ in network $G(V,E)$ is 
\begin{equation}
	U_{r, m, G} = \max_{(i,j)\in E} u_{ij}
\end{equation}

The $optimal routing$ in all the possible route $R$ for network $G$ is a routing which minimize the maximum utilization,
the minimum maximum utilization is called optimal utilization, can be represented by :
\begin{equation}
	OptU_{m, G} = \min_{r\in R} U_{r, m, G}
\end{equation}

The $performance ratio$ of a given routing $r$ on a given TM $m$ and a given network $G$ meaures how far from being 
optimal, it is defined as the maximum link utilization divided by optimal utilization on the $m$ and $G$, as following : 
\begin{equation}
	P(\{ r \},\{ m \}, G) = \frac{U_{r,m,G}}{OptU_{m,G}}
\end{equation}

We now extend the definition of performance ratio of a routing to be with respect to a set of TMs $M$. 
\begin{equation}
	P(\{ r \}, M, G) = \max_{m\in M} P(\{ r \}, \{ m \}, G)
\end{equation}

Obviously, the optimal routing in routing set $R$ for the set of TMs is a routing which minimize the extended 
performance ratio, such as :
\begin{equation}
	P(R, M, G) = \min_{r\in R} P(\{ r \}, M, G)
\end{equation}

I.E. the routing $r$ which arrive at the value of $P(R,M,G)$ is the most ``robust'' routing for the TM set $M$ 
in the network $G$, and if the $M$ range enough, we say that the ``robust'' routing is independent of specific TM.

But definition of ``robust'' above will not work well for next situation. Let us take a cycle network topology $C$ as 
an simple example, in which we should choose one link to close. Before link cutting, the $P(R, M, C)$ is 
approximate to 2, but no matter which link is chosen to close, the cycle network will change to a line network $L$. 
Obviously, the $P(R, M, L)$ will always equal to 1. It means that there is no difference from removing which link,
But the contradiction here is that, the choice for which link should be removed is really different because the 
links are not always the same with each other, such as their capacity.

The reason for the ``fake robust'' is that, the routing in the successor graph (i.e. $L$ in above example) become 
unique, the current routing always be the optimal routing. More generally, we should make a little modification
for the $performance ratio$ as the network self changes.

Let $G$ be the origin network, and the $G^* \in \Theta(G)$ be the successor network from $G$ after closing/removing
some link, we define $performance ratio between different graphs$ as the performance ratio of successor graph divide 
the optimal performance ratio of father graph, like :
\begin{equation}
	P(R^*, \{ m\}, G, G^*) = \min_{r \in R^*} \frac{U_{r,m,G^*}{OptU_{m,G}
\end{equation}
where $R^*$ is the routing set on network $G^*$

And question is that how to meaure a successor network topology is ``robust'' enough for the TM set $M$ when 
pruning is proceeding. We consider the worst situation, namely the successor network topology has its maximum
performance ratio when the TM is $m \in M$, described as following :
\begin{equation}
	P(R^*, M, G, G^*) = \max_{m \in M} P(R^*, \{ m \}, G, G^*)
\end{equation}

Now we can say that, if a successor network arrive the minimum performance ratio, it is the ``robust'' successor 
network graph. Formally, we define the performance ratio as $optimal successor performance ratio$ :
\begin{equation}
	P^{*}(M, G) = \min_{G^* \in \Theta(G)} P(R^*, M, G, G^*)
\end{equation}
where $M$ is the TM set, and $R^*$ is routing set. 

If the scope of $M$ is large engouh, the optimal successor network graph is alwo independent from specific TM. 

\subsection{Model Example}
As well, we will take an example to explain how to choose the link to close in our experiment. Figure shown as 
following, Three hostes: A, B, C, Three links repectively capacity of 3M, 4M and 2M. For simpleness, we suppose 
there are two traffic matrix: one trace 2M from A to B, and 1M from A to C; the other trace 1M from A to B, 
and 1M from A to C. For each traffic matrix, the optimal route is obvious, we will trace 2M from A to B across 
the lower link and trace the 1M from A to C across the upper one for the first traffic matrix, whose link 
utilization is 0.5. we will trace all the traffic acorss the lower link for the second traffic matrix, whose 
link utilization is 0.5 as well.

Now for some reason, we will choose one link to shut down for power saving without lose connection of the network. 
So two choice put before us, cut the upper link or the lower link. Let us take a little caluclation: when 
close the upper one, we should trace all the traffic acorss the lower link, as a result, in the first traffic 
matrix the link utilization is 0.75 and the other is 0.5; when close the lower link, we should trace all the 
traffic acorss the upper link, in the first traffic matrix the link utilization is 1 and the other is 0.667. 
So according to our theory, the $PerfGreen(T)$ should be 1.5 and the new network topology will be the one which 3M 
link is closed.


\section{Conclusion}
The conclusion goes here.

\begin{thebibliography}{1}

\bibitem{IEEEhowto:kopka}
H.~Kopka and P.~W. Daly, \emph{A Guide to \LaTeX}, 3rd~ed.\hskip 1em plus
  0.5em minus 0.4em\relax Harlow, England: Addison-Wesley, 1999.

\end{thebibliography}

% that's all folks
\end{document}


