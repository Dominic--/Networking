\documentclass[conference]{IEEEtran}

\usepackage{cite}

\ifCLASSINFOpdf
   \usepackage[pdftex]{graphicx}
  % declare the path(s) where your graphic files are
   \graphicspath{{./png/}}
  % and their extensions so you won't have to specify these with
  % every instance of \includegraphics
   \DeclareGraphicsExtensions{.png}
\else
  % or other class option (dvipsone, dvipdf, if not using dvips). graphicx
  % will default to the driver specified in the system graphics.cfg if no
  % driver is specified.
  % \usepackage[dvips]{graphicx}
  % declare the path(s) where your graphic files are
  % \graphicspath{{../eps/}}
  % and their extensions so you won't have to specify these with
  % every instance of \includegraphics
  % \DeclareGraphicsExtensions{.eps}
\fi

\ifCLASSOPTIONcompsoc
    \usepackage[caption=false,font=normalsize,labelfont=sf,textfont=sf]{subfig}
\else
    \usepackage[caption=false,font=footnotesize]{subfig}
\fi

    

% correct bad hyphenation here
\hyphenation{op-tical net-works semi-conduc-tor}

\begin{document}

\title{Robust Energy-Aware Routing with Uncertain Traffic Demands}


\author{\IEEEauthorblockN{Heng Lin}
\IEEEauthorblockA{Tsinghua University \\ henglin1991@gmail.com}
\and
\IEEEauthorblockN{Mingwei Xu}
\IEEEauthorblockA{Tsinghua University \\ xmw@cernet.edu.cn}
\and
\IEEEauthorblockN{Yuan Yang}
\IEEEauthorblockA{Tsinghua University \\ yyang@csnet1.cs.tsinghua.edu.cn}}


% make the title area
\maketitle

% As a general rule, do not put math, special symbols or citations
% in the abstract
\begin{abstract}
Energy conservation has become a major challenge to the Internet. In existing approaches, a part of line cards 
are switched into sleep mode for energy conservation, and the routing is configured carefully to balance energy saving 
and traffic engineering goals, such as the maximum link utilization ratio (MLUR). Typically, traffic demands are 
used as inputs, and routing is computed accordingly. However, accurate traffic matrices are difficult to obtain and 
are changing frequently. This makes the approaches difficult to implement. Further, the routing may shift 
frequently, and is not robust to sudden traffic changes.

In this paper, we propose a different approach that finds one energy-aware routing robust to a set of traffic 
matrices, particularly to arbitrary traffic demands. Such a routing without energy consideration is known as 
the demand-oblivious routing, and is well studied\cite{networking:oblivious}. However, the problem becomes much more challenging when energy 
conservation is involved. To overcome the challenges, we first define a new metric, namely oblivious performance 
ratio (OPR) with energy constraint, which reflects the MLUR distance from a routing to the optimal routing when 
certain energy conservation requirement is satisfied. We model the problem of minimizing the performance ratio, 
and analyze the lower and the upper bounds. Then, we propose Robust Energy-Aware Routing (REAR) to solve 
the problem in two phases. REAR select sleeping links based on extended robust link utilization or algebraic 
connectivity, and compute the routing based on a classical demand-oblivious routing algorithm. 
We evaluate our algorithms on real and synthetic topologies. The simulation results show that REAR can save 
19\% of line card power while the performance ratio is less than 34\%.
\end{abstract}

\IEEEpeerreviewmaketitle

\section{Introduction}
We aim to find the ``robust'' route for all possible traffic matrix, not only consider performance but also energy 
consumption in network.

\section{Motivation}
Suppose two hosts named host A and host B, and there are three links between 
them, respectively, capacity with 2M, 3M and 5M. Now demands come, with 1M from A to B. we regard the capacity 
as the power of the link, and the minimum maximum utilization of network links as a metric of the network performance. 
There are two directions for operating this example. One consider the min power consumption except for the 
utilization, it is obviously that we should close the larger power consuming links, such as the 5M and 3M links, and 
all the traffic go through the 2M link. In this way, the min max utilization of network is 0.5 and the power consuming 
is 2 units (means the power difference come from the link mainly). The other consider the power except for the 
utilization reversely, so we should split the 1M traffic to three parts, 0.2M across 2M link, 0.3M across 3M link and 
0.5M across 5M, consequently with a min max utilization of 0.1, but the power is 10 units however.

Two directions mentioned above both are extremely single-consideration. Previous researchers solve the problem more 
considerable, include ``GreenTE'' and ``a\%-green is engouh''. The former set a threshold of min max utilization, 
close links as many as possible to achive the most power saving. And the latter one set a destination of the power 
saving, calculate optimal route for get the min max utilization. Two work have their restriction, which both need a 
specific traffic matrix that as base of their optimization.

But the need of precise current traffic matrix shound be carefully checked. Although some researcher contribute to this 
area, the real precise traffic matrix still be a challenge. Take a step back, the dynamic of traffic matrix is more 
diffcult even if we obtain the precise current one. Futhermore, ISP will not want to change their route policy 
frequently, as it will result in other route failure possiblly. So our question is that : Is there exist a route both 
satisfy power and utilization requirement for any traffic matrix given?


David Applegate propose a method for obtain a route wihich is ``robust'' to variations in demands for a specific network topology. 


\section{Model}
We model the network as a undirected graph $G = (V, E)$, where $V$ is the set of vertices (i.e., routers and end hosts), 
and $E$ is the set of links (either link between routers or router and end-host). In graph $G$, two vertices $u$ and $v$
are called connected if $G$ contains a path between $u$ and $v$. Then We say graph $G$ is connected, if and only if 
arbitrary pair of vertices in the graph is connected.

Let $\theta(G) = \{ (V, E - \{ e \}) | e \in E \}$ denote the network set after closing/removing the link $e$ from
$G$. Then, choosing the connected graph from $\theta(G)$ to consist a new set, denoted by 
$\Theta(G) = \{G | G \in \theta(G) \ \& \& \ G\ is\ connected\}$. We call $\Theta(G)$ as successor of $G$.

A $traffic\ matrix$ (abbreviation as TM below) is the set of traffic between each Origin-Destination(OD) pair in 
network $G$, and a $routing$ specifies how traffic of each OD pair is routed across the network. Usually, there are 
multiple paths for each OD pair and each path routes a fraction of the traffic. Let $m$ denote the $traffic\ matrix$, 
which can be represented by a set of trinary group like $(a, b, d_{ab}$, where $a$ and $b$ is the origin and 
destination of pair respectively, $d_{ab}$ is the traffic demand of the OD pair. 

Let $r$ denote the $routing$ mentioned above, which is specified by a set of values $f_{ab}(i,j)$ that specifies the 
fraction of demand from $a$ to $b$ that is routed on the link $(i,j)$. So an OD pair contribute to the traffic of 
link $(i,j)$ is $d_{ab}f_{ab}(i,j)$, and all the traffic across link $(i,j)$ can be calculated as :
\begin{equation}
	\sum_{(a,b,d_{ab}\in m)} d_{ab}f_{ab}(i,j)
\end{equation}

Futhermore, we define the utilization of link as traffic acorss the link divide capacity of the link, as ;
\begin{equation}
	u_{ij} = \frac{\sum_{a,b} d_{ab}f_{ab}(i,j)}{cap_{ij}}
\end{equation}
where $cap_{ij}$ is the capacity of the link $(i,j)$.

A common metric for the performance of a given routing with respect to a certain TM is the $maximum\ link\ utilization$.
This is the maximum utilization of link over all ones, Formally, the maximum link utilization of a routing $r$ on 
TM $m$ in network $G(V,E)$ is 
\begin{equation}
	U_{r, m, G} = \max_{(i,j)\in E} u_{ij}
\end{equation}

The $optimal\ routing$ in all the possible route $R$ for network $G$ is a routing which minimize the maximum utilization,
the minimum maximum utilization is called optimal utilization, can be represented by :
\begin{equation}
	OptU_{m, G} = \min_{r\in R} U_{r, m, G}
\end{equation}

The $performance\ ratio$ of a given routing $r$ on a given TM $m$ and a given network $G$ meaures how far from being 
optimal, it is defined as the maximum link utilization divided by optimal utilization on the $m$ and $G$, as following : 
\begin{equation}
	P(\{ r \},\{ m \}, G) = \frac{U_{r,m,G}}{OptU_{m,G}}
\end{equation}

We now extend the definition of performance ratio of a routing to be with respect to a set of TMs $M$. 
\begin{equation}
	P(\{ r \}, M, G) = \max_{m\in M} P(\{ r \}, \{ m \}, G)
\end{equation}

Obviously, the optimal routing in routing set $R$ for the set of TMs is a routing which minimize the extended 
performance ratio, such as :
\begin{equation}
	P(R, M, G) = \min_{r\in R} P(\{ r \}, M, G)
\end{equation}

I.E. the routing $r$ which arrive at the value of $P(R,M,G)$ is the most ``robust'' routing for the TM set $M$ 
in the network $G$, and if the $M$ range enough, we say that the ``robust'' routing is independent of specific TM.

But definition of ``robust'' above will not work well for next situation. Let us take a cycle network topology $C$ as 
an simple example, in which we should choose one link to close. Before link cutting, the $P(R, M, C)$ is 
approximate to 2, but no matter which link is chosen to close, the cycle network will change to a line network $L$. 
Obviously, the $P(R, M, L)$ will always equal to 1. It means that there is no difference from removing which link,
But the contradiction here is that, the choice for which link should be removed is really different because the 
links are not always the same with each other, such as their capacity.

The reason for the ``fake robust'' is that, the routing in the successor graph (i.e. $L$ in above example) become 
unique, the current routing always be the optimal routing. More generally, we should make a little modification
for the $performance\ ratio$ as the network self changes.

Let $G$ be the origin network, and the $G^* \in \Theta(G)$ be the successor network from $G$ after closing/removing
some link, we define $performance\ ratio\ between\ different\ graphs$ as the performance ratio of successor graph divide 
the optimal performance ratio of father graph, like :
\begin{equation}
	P(R^*, \{ m\}, G, G^*) = \min_{r \in R^*} \frac{U_{r,m,G^*}}{OptU_{m,G}}
\end{equation}
where $R^*$ is the routing set on network $G^*$.

And question is that how to meaure a successor network topology is ``robust'' enough for the TM set $M$ when 
pruning is proceeding. We consider the worst situation, namely the successor network topology has its maximum
performance ratio when the TM is $m \in M$, described as following :
\begin{equation}
	P(R^*, M, G, G^*) = \max_{m \in M} P(R^*, \{ m \}, G, G^*)
\end{equation}

Now we can say that, if a successor network arrive the minimum performance ratio, it is the ``robust'' successor 
network graph. Formally, we define the performance ratio as $optimal\ successor\ performance\ ratio$ :
\begin{equation}
	P^{*}(M, G) = \min_{G^* \in \Theta(G)} P(R^*, M, G, G^*)
\end{equation}
where $M$ is the TM set, and $R^*$ is routing set. 

If the scope of $M$ is large engouh, the optimal successor network graph is alwo independent from specific TM. 

\subsection{Model Example}
We will take an example to explain how to choose the link to close in our experiment.
Three hostes include : A, B, C, Three links are with repectively capacity of 3M, 4M and 2M. For simpleness, we suppose 
there are two TM : ${(A,B,2M), (A,C,1M)}$ and ${(A,B,1M), (A,C,1M)}$. For each traffic matrix, the optimal route is 
obvious, we will trace 2M from A to B across the lower link and trace the 1M from A to C across the upper one 
for the first traffic matrix, whose maximum link utilization is 0.5. we will trace all the traffic acorss the
lower link for the second traffic matrix, whose maximum link utilization is 0.5 as well.


\begin{figure}[!t]
\centering
\vspace*{0.1in}
\includegraphics[width=8cm]{3-nodes-example}
\caption{Model Example}
\label{label}
\vspace*{0.1in}
\end{figure}


Now for some reason, we will choose one link to shut down for power saving without lose connection of the network. 
There are two choice, remove either the upper link or the lower link. Let us take a little calculation: when 
remove the upper one, we should change all the traffic acorss the lower link, as a result, in the first  
TM the maximum link utilization is 0.75 and the second is 0.5; when close the lower link, we should trace all the 
traffic acorss the upper link, in the first TM the link utilization is 1 and the other is 0.667. 
So according to our theory, the $P^{*}(M, G)$ should be 1.5 and the optimal successor network topology will be
the one which 3M link is closed.

\section{Algorithm}
In our paper, Robust Energy-Aware Routing (REAR) algorithm works in two phases. Firstly, REAR select links 
should be sleeping from the origin topology based on extended robust link utilization or algebraic connectivity, 
then compute the robust routing by demand-oblivious routing algorithm. There are two metrics
for selecting removed links, our two-phase algorithm is flexible because we can replace the metric with whatever
metric which will be proposed in the future.


\subsection{Algebraic Connectivity}
Network topology is represented by $G = (V, E)$ as mentioned in Model section, where $V$ is the set of vertices
and $E$ is the set of links. We say $A(G)$ is the Adjacency Matrix of graph $G$, that include information for which
vertices of the graph are adjacent to which other vertices. $A(G)$ is a $N \times N$ matrix, where
$N = |V|$ and non-diagonal entry $a_{ij}$ equal to the number of edges from vertex $i$ to vertex $j$, in this paper,
$a_{ij}$ always be 1 if $(i,j) \in E$ otherwise 0. And we stipulate the diagonal element $a_{ii}$ be 0.


We say $D(G)$ is the Degree Matrix of graph $G$, which is a diagonal matrix and diagonal entry $d_{ii}$ denote 
the degree of node $i$. It is obvious that there is $d_{ii} == \sum_{j} a_{ij}$.


Then we define Laplacian Matrix $L(G)$ of graph $G$ as the difference between Degree Matrix and Adjacency Matrix :
\begin{equation}
	L(G) = D(G) - A(G)
\end{equation}
where $D(G)$ is the Degree Matrix and $A(G)$ is the Adjacency Matrix.


In the mathematical field of graph theory, the number of eigenvalues equal to 0 is the number of connected 
componements of $G$, so the smallest eigenvalue always be 0 in arbitrary graph . And we call the second smallest one as
algebraic connectivity, which is greater than 0 if and only if graph $G$ is connected. Further more,  
it measures the connectivity and stability of graph, the greater of which, the more connective of graph;
and it is a metric of average distance between any two vertices of graph $G$. We call the algebraic connectivity as
$\lambda_2(G)$.


Topology always have different algebraic connectivity values, it is clear that when one link is added or 
removed from the graph, the algebraic connectivity value changes accordingly. And we say the changed value
is the impact of this link on the graph. Supposed we sleep link $l$ from graph $G$, new graph is described as $G^*$, 
we defined the impact of link $l$ as :
\begin{equation}
	\Delta_l = \lambda_2(G) - \lambda_2(G^*)
\end{equation}
where $\lambda_2(G)$ and $\lambda_2(G^*)$ is algebraic connectivity of $G$ and $G^*$ respectly.


Clearly, $\Delta_l$ is always greater than 0, because graph will always lose connectivity when link is removed.
Further more, some link will play a more important role in the connectivity of graph, such as the backbone link 
of network topology. And we say a link $l$ affect more if $\Delta_l$ is greater.

\subsection{Extended Robust Link Utilization}
The demand-oblivious routing is well studied before, which means that this routing is not always the best one for 
specific $TM$, but good enough for a range of $TMs$. we mentioned the definition of link utilization above when 
demand and routing are determined, as demand become oblivious, the definition is also not suitable. So we define
a notation named $extended\ robust\ link\ utilization$ replace the normal link utilization like :
\begin{equation}
	u^e_{ij} = \frac {\sum_{a,b}f_{ab}(i,j)} {cap_{ij}}
\end{equation}


Proposing this definition for two consideration: firstyly, the more flows across the link, the greater the $u^e_{ij}$ is;
secondly, the more fraction of traffic in one flow across the link, the greater the $u^e_{ij}$ is. The link with greater
extended robust link utilization is also more important than others in a sense. Although there is no demand here, we mean
there is more probability that this link have greater link utilization when specific demand come. Similarly, we define 
the impact of link $l$ as :
\begin{equation}
    \Delta_l = u^e_{l}
\end{equation}

And there is the same conclusion that a link $l$ affect more if $\Delta_l$ is greater.


\subsection{Algorithm Phase One}
REAR sleep as many links as possible without losing much connectivity or transportation of graph for different metrics. 
Originly,  we should calculate impact of all the links and sleep the lowest one from the graph, then repeat calculate and remove
process until arrive some specific threhold. Obviously, it is NP-Hard, following is a heuristic algorithm.


For the origin graph, we calculate the impact of every link as $\Delta_{l_i}$ , and then sort these values
from small to big, output ordered list denoted as $\Gamma$: 
\begin{equation}
	\Gamma = \{..., l_i, ..., l_j, ...\}
\end{equation}
where $\Delta_{l_i} < \Delta_{l_j}$.


Pay attention we only compute the link impact once at the begining of algorihtm, and the ordered list $\Gamma$ show 
the order of `importance` among links in the graph. 


Now we begin selecting which links should be sleep.
We denote the set of the sleeping links as $S$, and the output of this phase is final graph $G^* = (V, E-S)$. We set 
$S = \emptyset$, and repeat our selecting process, each iteration we select one link, remove it from $\Gamma$ and put it into $S$. 
In iteration $i$, algorithm scan links as the order in $\Gamma$, we try to
remove this link from the graph to check if the graph is still connected and the energy conservation have not arrive the threshold.
If so, we select this one then go next iteration.
Otherwise choose the next link from $\Gamma$ for trying to remove. Algorithm
stop until all the links in $\Gamma$ is tried but no one is satisfied with both connectivity and power threhold.


So before going ahead our algorithm, there is another thing we should done, how to measure the power of graph.
We simple take an power model from `Green TE` showed in Table xx, and defined the difference of power consumption 
between two graphs as:
\begin{equation}
	diff_p = \rho(G_{S, l}^o) / \rho(G^o) * 100
\end{equation}
where $\rho(G_{S, l}^o)$ is the power consumption of the final graph, when the links set $S$ and 
link $l$ are both removed from the origin graph $G^o$, and the $\rho(G^o)$ is the power consumption of 
the origin graph.


If we set $diff_p$ valued 90\%, it means that whenever we try to remove the link $l$ from the origin graph in 
iteration, the power consumption should never lower than the 90\% of origin one. In another words, the output 
of this phase protect as much connectivity and transportation as possible. Following is our implementation: 


\begin{table}[!th]
\begin{tabular}{ll}
\hline
\textbf{Algorithm REAR : Phase One}\\
\hline
$\:\:$\textbf{Input:} $G(V, E)$, $threhold$;\\
$\:\:$\textbf{Output:} $S$ in which links should be switched off;\\
$\quad\qquad\quad$ $G(V, E-S)$ which is the final network topology;\\
$\:\:$1:\ \textbf{for} {each link $l$ in $E$}\\
$\:\:$2:\quad\ $G^* \leftarrow G(V, E-\{l\})$;\\
$\:\:$3:\quad\ $\Gamma[l] \leftarrow \Delta_l \leftarrow \lambda_2(G) - \lambda_2(G^*)$;\\
$\:\:$4:\ Resort $\Gamma$ in increasing order based on $\Delta_l$;\\
$\:\:$5:\ $S \leftarrow \emptyset$, $goon \leftarrow true$;\\
$\:\:$6:\ \textbf{while} {$goon$}\\
$\:\:$7:\quad\  $goon \leftarrow false$;\\
$\:\:$8:\quad\ \textbf{for} {each link $l$ in $\Gamma - S$}\\
$\:\:$9:\quad\ \quad\ \textbf{if} $G_{S,l}$ is connected and $\rho(G_{S,l})/\rho(G)>threhold$\\
$\:\:$10:\quad\ \quad\ \quad\ $S \leftarrow S \cup \{l\}$;\\
$\:\:$11:\quad\ \quad\ \quad\ $goon \leftarrow true$;\\
$\:\:$12:\quad\ \quad\ \quad\ \textbf{break};\\
$\:\:$13:\ \textbf{return} $S, G(V, E-S)$;\\
\hline
\end{tabular}
\end{table}



\subsection{Algorihtm Phase Two}
Once we get the output network topology from the first phase based on either metric, it is time to compute
the robust routing. On one hand, computation process may cost too much time if we directly calculate the robust routing in 
the final graph; on the other hand, the robust routing based on the final graph may not be the best one.
There is a heuristic algorithm
based on the demand-oblivious routing on the origin graph, which should be obtained at first.
then adjust routing in details according to the links we switched off.


The demand-oblivious routing can be computed by a single LP with $O(mn^2)$ variables and $O(nm^2)$ constraints[1] :
\begin{table}[!th]
\begin{tabular}{ll}
$\:\:$\quad\quad\quad\quad\ \textbf{min} $r$ \\
$\:\:$\quad\quad\quad\quad\ $f_{ij}(e)$ is a routing \\
$\:\:$\quad\quad\quad\quad\ $\forall$ links $l$: $\sum_m cap(m)t(l,m) \le r$ \\
$\:\:$\quad\quad\quad\quad\ $\forall$ links $l$, $\forall$ pairs $i \rightarrow j$: \\
$\:\:$\quad\quad\quad\quad\quad\quad\ $f_{ij}(l)/cap(l) \le p_l(i,j)$ \\ 
$\:\:$\quad\quad\quad\quad\ $\forall$ links $l$, $\forall$ nodes $i$, $\forall$ edges $e = j \rightarrow k$: \\
$\:\:$\quad\quad\quad\quad\quad\quad\ $\pi(l, link-of(e)) + p_l(i,j) - p_l(i,k) \ge 0$ \\
$\:\:$\quad\quad\quad\quad\ $\forall$ links $l$, $m$: $\pi(l, m) \ge 0$ \\
$\:\:$\quad\quad\quad\quad\ $\forall$ links $l$, $\forall$ nodes $i$: $p_l(i,i) = 0$ \\
$\:\:$\quad\quad\quad\quad\ $\forall$ links $l$, $\forall$ nodes $i$, $j$: $p_l(i,j) \ge 0$ \\
\end{tabular}
\end{table}


where the $cap(l)$ is the capacity of link $l$; 
and $\pi(l,m)$ is the weights for every pair of links $l$, $m$; and the variables $p_l(i,j)$ for each link $l$ 
and OD pair $i$, $j$ is the length of the shortest path from $i$ to $j$ according to the link weights $\pi(l, m)$.


The routing we get indicate how to arrive at destination node from source node for every OD pair in the origin topology.
What is different is that, the flow can be splited in the routing, i.e. there may be two paths ($path_1$, $path_2$)
both from source node $s$ to destination node $d$, and the optimal obilious routing trace 70\% traffic on 
$path_1$ and left on $path_2$. Although splitting flow is hard handled, we take a transformation for the case like that:
when an flow is coming, there is 70\% probability we trace it on $path_1$, otherwise $path_2$. This is easily implementated 
in real world.


Because all the routing is based on the origin topology, when some links are switched off, we must adjust the routing 
as well. Supposed there are some paths between two vertices $s$ and $d$, maybe one link in some paths be removed, 
and these paths become not reachable any more, we should adjust the traffic in these paths to other paths. Of course,
we can not put the whole traffic on another path, this will make some links of the path congested, so we should split the traffic 
to some paths `averagely`, in the sense of extended robust link utilization. Similarly to link utilization, we define the extended
robust link utilization as the maximum extended robust link utilization of links.


Take pair $(s, d)$ for example, routing includes paths from $s$ to $d$, such as $p_1, p_2, ... , p_n$. While only $p_1$ trace 
the removed link, and of course it will be unreachable. We split the traffic of $p_1$, and put them on other paths to make 
the extend robust link utilization of paths almost closely.


And this is our implementation:
\begin{table}[!th]
\begin{tabular}{ll}
\hline
\textbf{Algorithm REAR : Phase Two}\\
\hline
$\:\:$\textbf{Input:} $G(V,E)$ which is the origin topology;\\
$\quad\quad\ \ \ $ $S$ which is the switch-off links set generated by Phase One;\\
$\quad\quad\ \ \ $ $R$ which is the Robust Routing on origin topology;\\
$\:\:$\textbf{Output:} $Routing$ Robust Energy-Aware Routing on new topology\\
$\:\:$\ 1:\ $G^* \leftarrow G(V, E-S)$;\\
$\:\:$\ 2:\ \textbf{for} {each link $l$ in $S$}\\
$\:\:$\ 3:\quad\ \textbf{for} {each $s,d,paths$ in $R$}\\
$\:\:$\ 4:\quad\ \quad\ $traffic \leftarrow$ 0;\\
$\:\:$\ 5:\quad\ \quad\ \textbf{for} {each $path$ in $paths$}\\
$\:\:$\ 6:\quad\ \quad\ \quad\ \textbf{if} {$link$ in $path$}\\
$\:\:$\ 7:\quad\ \quad\ \quad\ \quad\ $traffic \leftarrow$ $traffic$ + $path$.$traffic$; \\
$\:\:$\ 8:\quad\ \quad\ \quad\ \quad\ $paths$.remove($path$); \\
$\:\:$\ 9:\quad\ \quad\ $yen\_paths \leftarrow$ yens\_algorithm($G^*$, $s$,$d$);\\
$\:\:$10:\quad\ \quad\ \textbf{while} {$traffic$ != 0}\\
$\:\:$11:\quad\ \quad\ \quad\ sort\_paths\_by\_extend\_robust\_link\_utilization($yen\_paths$)\\
$\:\:$12:\quad\ \quad\ \quad\ $yen\_paths$[0].$traffic$ = $traffic$ / N\\
$\:\:$13:\quad\ \quad\ \quad\ $traffic$ = $traffic$ - $traffic$ / N\\
$\:\:$14:\quad\ \quad\ \quad\ $paths$.add($yen\_paths$[0])\\
$\:\:$15:\quad\ \quad\ $paths$.merge();\\
$\:\:$16:\ \textbf{return} $R$;\\
\hline
\end{tabular}
\end{table}


\section{Samples and Proof}
For showing obilious performance ratio with energy constraint really make a difference in the switching off link 
process, we will explain it in two simple topologies, cliques and cycles. On the other hand, we also say that
there will be an upper bound for obilious performance ratio with energy constraint, and get the bound according
to the topology. For simpleness, we refer the energy constraint to the quantity of removed links.

\begin{figure}[!t]
\centering
\vspace*{0.1in}
\subfloat[Circle]{\includegraphics[width=4cm]{circle}
\label{subfiga}}
\hfill
\subfloat[Clique]{\includegraphics[width=4cm]{clique}
\label{subfigb}}
\caption{Circle and Clique topology of 6 nodes}
\vspace*{0.1in}
\end{figure}


\subsection{Circles}
Cycles topology connect all the nodes by a cycle, any node in which has two links joined. Figure 1 show the six
nodes cycles topology, let nodes named from $A$ to $F$ and links named from $a$ to $f$ in the figure.
Now suppose a traffic in the topology then calculate the optimal routing for it,
let $D_a$ and $C_a$ be the demand and capacity of the link, without loss of generality, we say the link $d$ is the 
bottleneck, i.e. with the maximum link utilization. There are two solutions for us, switching off the link $d$,
or the other link, such as link $f$.


For case one, when we switch off the link $d$, the origin traffic pass the link will be redirected. The worst situation
is that all the traffic on the link $d$ will be put to the other path : $D \leftarrow C \leftarrow B \leftarrow A \leftarrow 
F \leftarrow E$. After traffic redirection, the bottleneck link may become the link $e$, according to our definition of
extended robust link utilization:

\begin{equation}
    \frac {\frac{D_e + D_d} {C_e}} {\frac {D_d}{C_d}} = (1 + \frac{D_e}{D_d}) \frac{C_d}{C_e}
\end{equation}

Because the link utilization of link $d$ is the maximum one, we have :

\begin{equation}
    \frac{D_d}{C_d} > \frac{D_e}{C_e} 
\end{equation}
\begin{equation}
    \frac{D_e * C_d}{C_e * D_d} < 1
\end{equation}

so we have the extended robust link utilization in case one :

\begin{equation}
    (1 + \frac{D_e}{D_d}) \frac{C_d}{C_e} < \frac{C_d}{C_e} + 1
\end{equation}


For case two, similar to case one, all the traffic will be redirected on the other path between $F$ and $A$. There are two 
sub cases after traffic redirection, the link $D$ is still the bottleneck or the link $E$ become the new bottleneck. In sub 
case one:

\begin{equation}
    \frac{\frac{D_d + D_f}{C_d}}{\frac{D_d}{C_d}} = 1 + \frac{D_f}{D_d}
\end{equation}

Because the link utilization of link $d$ is the maximum one in origin topology, we have as:

\begin{equation}
    \frac{D_d}{C_d} > \frac{D_f}{C_f} 
\end{equation}
\begin{equation}
    \frac{D_f}{D_d} < \frac{C_d}{C_f} 
\end{equation}

So we get the extended robust link utilization in this case :
\begin{equation}
    (1 + \frac{D_f}{D_d}) < \frac{C_d}{C_f} + 1
\end{equation}

In sub case two, we get the new extended robust link utilization in link $E$:

\begin{equation}
    \frac{\frac{D_e + D_d}{C_e}}{\frac{D_d}{C_d}} = \frac{D_e * C_d}{D_d * C_e} * \frac{C_d * D_f}{C_e * D_e}
\end{equation}

Similarly as mentioned above, we get equations :

\begin{equation}
    \frac{D_d}{C_d} > \frac{D_e}{C_e} => \frac{D_e * C_d}{C_e * D_d} < 1 
\end{equation}
\begin{equation}
    \frac{D_d}{C_d} > \frac{D_f}{C_f} => \frac{D_f * C_d}{D_d} < C_f
\end{equation}

So we get the extended robust link utilization:

\begin{equation}
    \frac{D_e * C_d}{D_d * C_e} + \frac{C_d * D_f}{C_e * D_d} < 1 + \frac{C_f}{C_e}
\end{equation}

As a result, the upper bound of extended robust link utilization must be :

\begin{equation}
    Max\{1 + \frac{C_d}{C_e}, 1 + \frac{C_d}{C_f}, 1 + \frac{C_f}{C_e}\}
\end{equation}

Particularly, if all the links have the same capacity, the upper bound equal to 2 when switch off one link. Otherwise, the 
upper bound only dependent to the greatest ratio of link capacity.
 
\subsection{Cliques}
In clique topology, all the nodes connect to each other, it means that situation become more difficult because the traffic
can be adjusted to multi paths rather than one path in circle. And we will see that splitting traffic on multi paths is 
always better than put all the traffic to one, so if we get the upper bound of the worst case, it must be the upper bound
of other case. Similarly to circle topology, we say the link $a$ is the bottleneck, and we will switch off it or other link
like $b$.

In case one, after we switch off link $a$, the new bottleneck link must be the affected link, the link contains the origin 
traffic on link $a$. Otherwise if the link $b$ is the bottleneck, but we adjust the traffic on other links, it means that
we can do the same thing in the origin topology, and the bottleneck become link $b$ rather than $a$. It is conflict with our
assumption. So we have the extended robust link utilization as :

\begin{equation}
    \frac {\frac{D_b + D_a}{C_b}}{\frac{D_a}{C_a}} = (1+\frac{D_b}{D_a}) \frac{C_a}{C_b}
\end{equation}

Because the link $a$ is the bottleneck link with the maximum link utilization, like what we do above, we can get 

\begin{equation}
    (1+\frac{D_b}{D_a}) \frac{C_a}{C_b} < 1 + \frac{C_a}{C_b}
\end{equation}

In case two, the bottleneck may still be link $a$ or the new link $c$. If it is link $a$, and no new traffic across it,
the extended robust link utilization equal to 1 obiliously. Or we adjust the new traffic $D_b$ on it, we can get the
extended robust link utilization:

\begin{equation}
    \frac{\frac{D_a + D_b}{C_a}}{\frac{D_a}{C_a}} = 1 + \frac{D_b}{D_a} < 1 + \frac{C_b}{C_a}
\end{equation}

If the bottleneck become new link $c$, similar to our proof of circle topology, we can get extended robust link utilization:
\begin{equation}
    \frac{\frac{D_c + D_b}{C_c}}{\frac{D_a}{C_a}} = \frac{D_c * C_a}{D_a * C_c} + \frac{D_b * C_a}{D_a * C_c} < 1 + \frac{C_b}{C_c}
\end{equation}

So we obtain the same conclusion as circle topology, the extended robust link utilization must be :

\begin{equation}
    Max\{1 + \frac{C_a}{C_b}, 1 + \frac{C_b}{C_a}, 1 + \frac{C_b}{C_c}\}
\end{equation}

\subsection{Other Topology}
We can also see that, when we switch off different link, it really matters the performance which dependent on not only the 
demand or traffic, but also the capacity of links. Particularly, the upper bound is just related with the capacity of links.
And from our proof process, there is no special limitation for specific topology, so we can get the similar conclusion 
on other topology.


\section{Experiments and Results}
We implemented our algorithm on real world topology, including Abilene, Geant and Cernet2, and we use Gravity model to generate
random traffic matrix. Because we take an attribute $w$ to expand the traffic range from $1/w$ to $w$, when the $w$ increased the
traffic varied in wider range. Particularly, when the $w$ limit extremity, we say the traffic matrix is really random, and our 
algorithm is irrelevant with traffic.

\begin{figure}[!t]
\centering
\vspace*{0.1in}
\subfloat{\includegraphics[width=4cm]{opr_with_remove_links}}
\subfloat{\includegraphics[width=4cm]{opr_with_power_saving}}
\caption{OPR with Power Saving}
\vspace*{0.1in}
\end{figure}

In Figure, we take $w$ as 1.5 and observe that: when we switch off more links, the OPR is increased as well. 
It is obvious that when the topology
lose connectivity by removing links, the worst case of robust routing must be far away from the optimal case in origin topology. And
in Geant and Abilene, even in the origin topology, the OPR value still not be 1. It is reasonable, because our robust routing 
is used for a range of traffic, however the optimal routing is calculated for the specific traffic matrix. But fortunately, our robust
routing is only $24\%$ greater than the optimal routing even in its worst case of Geant. And we also see that the curve become 
cliffy in the end, because when we remove the links the toplogy lose too much connectivity and there alwasy one path between two vertices.
Obviously, the performance will become worse in the end. Lastly we say that although the OPR relation really different in different 
topology, but the curve still be similar with each other, they are exponetial curve correspond with their vertices amount.


On the other hand, we may concern how is the OPR varifying when the power saving achived besides the amount of removed links.
Of course the more we removed the links, the more power we saved, but how to quantify the power of one link is difficult. Green
TE proposed the simple power model for topology, which can be represented in Table x:

\begin{table}[!t]
\renewcommand{\arraystretch}{1}
\caption{Green TE Power Model}
\label{power model}
\centering
\begin{tabular}{|c|c|c|}
\hline
\bfseries Line-Card & \bfseries Speed(Mbps) & \bfseries Power(Watts) \\
\hline
1-Port OC3 & 155.52 & 60 \\
\hline
8-Port OC3 & 1244.16 & 100 \\
\hline
1-Port OC48 & 2488.32 & 140 \\
\hline
1-Port OC192 & 9953.28 & 174 \\
\hline
\end{tabular}
\end{table}

We compute the power saving ratio as the total power of the removed links over the total power of all the links. And we can
save 19\% power with only 34\% worse than the worst case for all the traffic in Abilene topology. In Geant and Cernet2, 
the OPR is little higher. Also we can see that two ways for removing links perform almost the same at the begin in Geant,
and varified at the end. It shows that we may find another better way for removing the links in the future.


For computing the worst case, we generate 1000 traffic matrix for every topoloy and the $w$ with the Gravity Model.
We show the each result in the Figure x and x when we remove the first 4 links in each topology. In Abilene, the result 
is more tighter than Geant. Comparsion in two ways, we can see that in Abilene the AC method is worse than the ERLU one,
and the downmost zone is the same in two ways, because the robust routing is same in the origin topology.

\begin{figure}[!t]
\centering
\vspace*{0.1in}
\subfloat{\includegraphics[width=4cm]{exp2_nosort_abilene}}
\subfloat{\includegraphics[width=4cm]{exp2_nosort_geant}}
\caption{Nosort: (a). Abilene (b).Geant}
\vspace*{0.1in}
\end{figure}


\begin{figure}[!t]
\centering
\vspace*{0.1in}
\subfloat{\includegraphics[width=4cm]{exp2_sort_abilene}}
\subfloat{\includegraphics[width=4cm]{exp2_sort_geant}}
\caption{Sort: (a). Abilene (b).Geant}
\vspace*{0.1in}
\end{figure}

From Figure xx and Figure xx, we sort the 1000 TMs by their OPR value, and see that most of the TMs are quite average. 
It is said that, our metric OPR is always not be achived for most traffic matrix.


We know when we compute the robust routing for specific topology, the $w$ is very important. When the $w$ limit extremely,
our robust routing is said without knowledge of traffic matrix. Figure xx, show the OPR varified when $w$ changes. The greater
the $w$ is, the greater the OPR is. It is obvious because when the $w$ is greater, the traffic matrix is random in more wider 
range, and our OPR is the worst case of all the traffic matrix.

\begin{figure}
\centering
\vspace*{0.1in}
\includegraphics[width=8cm,height=4cm]{exp3_w}
\caption{OPR with w relationship}
\vspace*{0.1in}
\end{figure}


\section{Related Work}
related work.


\section{Conclusion}
The conclusion goes here.

\begin{thebibliography}{1}

\bibitem{networking:greente}
M.Zhang, C.Yi, B.Liu and B.Zhang, "GreenTE: Power-Aware Traffic Engineering".

\bibitem{networking:oblivious}
D.Applegate and E.Cohen, "Making Intra-Domain Routing Robust to Changing and Uncertain Traffic Demands: Understanding Fundamental Tradeoffs".

\end{thebibliography}


% that's all folks
\end{document}



























