\documentclass[conference]{IEEEtran}

\ifCLASSINFOpdf
  % \usepackage[pdftex]{graphicx}
  % declare the path(s) where your graphic files are
  % \graphicspath{{../pdf/}{../jpeg/}}
  % and their extensions so you won't have to specify these with
  % every instance of \includegraphics
  % \DeclareGraphicsExtensions{.pdf,.jpeg,.png}
\else
  % or other class option (dvipsone, dvipdf, if not using dvips). graphicx
  % will default to the driver specified in the system graphics.cfg if no
  % driver is specified.
  % \usepackage[dvips]{graphicx}
  % declare the path(s) where your graphic files are
  % \graphicspath{{../eps/}}
  % and their extensions so you won't have to specify these with
  % every instance of \includegraphics
  % \DeclareGraphicsExtensions{.eps}
\fi

% correct bad hyphenation here
\hyphenation{op-tical net-works semi-conduc-tor}


\begin{document}

\title{Draft for Networking Paper}


\author{\IEEEauthorblockN{Michael Shell}
\IEEEauthorblockA{Tsinghua University}
\and
\IEEEauthorblockN{Homer Simpson}
\IEEEauthorblockA{Tsinghua University}
\and
\IEEEauthorblockN{James Kirk}
\IEEEauthorblockA{Tsinghua University}}


% make the title area
\maketitle

% As a general rule, do not put math, special symbols or citations
% in the abstract
\begin{abstract}
The abstract goes here.
\end{abstract}

\IEEEpeerreviewmaketitle

\section{Introduction}
Introduction


\section{Notes}
With network developed, it is consist of more and more end-hosts, routers, switches, the topology aslo become more
complex than ever before. As a result, we not only consider the utilization but also the energe consuming of the network.

Things seem to be not so obviously simple. Suppose two hosts named host A and host B, and there are three links between them,
respectively, capacity with 2M, 3M and 5M. Now traffic matrix comes, with 1M from A to B. we regard the capacity as the power
of the link, and the min max utilization of network links as a metric of the network performance. There are two 
directions for operating this demo network. One consider the min power consumption except for the utilization, 
it is obviously that we should close the larger power consuming links, such as the 5M and 3M links, and all the traffic 
go through the 2M link. In this way, the min max utilization of network is 0.5 and the power consuming is 2 units 
(means the power difference come from the link mainly). The other consider the power except for the utilization reversely,
so we should split the 1M traffic to three parts, 0.2M across 2M link, 0.3M across 3M link and 0.5M across 5M, consequently with 
a min max utilization of 0.1, but the power is 10 units however.

Two directions mentioned above both are extremely single-consideration. Previous researchers solve the problem more considerable,
include ``GreenTE'' and ``a\%-green is engouh''. The former set a threshold of min max utilization, close links as many 
as possible to achive the most power saving. And the latter one set a destination of the power saving, calculate optimal route 
for get the min max utilization. Two work have their restriction, which both need a specific traffic matrix that as base of 
their optimization.

But the need of precise current traffic matrix shound be carefully checked. Although some researcher contribute to this area,
the real precise traffic matrix still be a challenge. Take a step back, the dynamic of traffic matrix is more diffcult even
if we obtain the precise current one. Futhermore, ISP will not want to change their route policy frequently, as it will result
in other route failure possiblly. So our question is that : Is there exist a route both satisfy power and utilization 
requirement for any traffic matrix given?

David Applegate propose a method for obtain a route wihich is ``robust'' to variations in demands for a specific network 
topology. We define $d_{ab}$ as the demand from $a$ to $b$, routing can be specified by a set of values $f_{ab}(i,j)$ which 
means the fraction demand from $a$ to $b$ routed on the link $(i,j)$. Obviously, the $d_{ab}$ contribute to the link $(i,j)$ is 
$d_{ab}f_{ab}(i,j)$. We also defined $cap_{ij}$ as the capacity of the link $(i,j)$; 

Formally, the max link utilization of a routing $r$ on traffic matrix $tm$ in network topology $T$ can be described as following:
\begin{equation}
U_{r,tm, T} = \max_{(i,j)\in linksof(T)} \sum_{a,b\in tm} \frac{d_{ab}f_{ab}(i,j)}{cap_{ij}}
\end{equation}

The routing which own the min max link utilization is the optimal one in the possible routing set $R$ of network topology $T$,
it can be represented by :
\begin{equation}
OptU_{tm, T} = \min_{r\in R} U_{r,tm, T}
\end{equation}

Now we define what is ``robust'', before this we introdunce the ``distance'' between the current routing $r$ and the optimal routing
in the some traffic matrix $tm$ of network topology $T$, as following :
\begin{equation}
Perf(\{ r \},\{ tm \}, T) = \frac{U_{r,tm,T}}{OptU_{tm, T}}
\end{equation}

and the ``robust'' routing satisfied the min max distance for all possible traffic matrix, 
\begin{equation}
Perf(\{ r \}, T) = \max_{tm\in TM} Perf(\{ r \}, \{ tm \}, T)
\end{equation}

\begin{equation}
Perf(R, TM, T) = \min_{r\in R} Perf(\{ r \}, T)
\end{equation}

i.e. the routing $r$ achive the $Perf(R,TM)$ represent the most ``robust'' routing for the traffic matrix set ``TM''. Back to our
origin question, we change another formulation : Is there exist a robust routing both satisfy power and utilization requirement?
and if does, how to obtain it?

Definition for the ``robust'' will failed in our power and utilization model. Let us take a cycle network topology as an simple example,
in which we should choose one link to close. Before link cutting, the $Perf(R, TM, CYCLE)$ is approximate to 2, but no matter which 
link is chosen to close, the $Perf(R, TM, LINE)$ will change to 1. The reason for the ``fake robust'' is that the route in the 
after-link-cutting topology become unique, the current route always equal the optimal route. More generally, when taking the green in
consideration, with the link cutting process proceeding, the previous definition of ``robust'' always go down because the network become
more and more monotonous.

So we make a little modification for the $Perf$, in our situation we will change the status of link frequently, which means
the topology of network also changed all the time, so the $Perf$ is related to the difference network topology, 
let $T_{ori}$ be the origin network, and the $T_{new}$ be the nre network after closing some link, we redefine the $Perf$ as:
\begin{equation}
Perf(\{ tm\}, T_{ori}, T_{new}) = \min_{r \in routeof(T_{new})} \frac{U_{r,tm,T_{new}}}{OptU_{tm,T_{ori}}}
\end{equation}

When close the some link, we choose the min max $Perf$ :
\begin{equation}
PerfGreen(T) = \min_{T_{new}\in from(T)} \max_{tm \in TM} Perf(\{ tm \}, T_{ori}, T_{new})
\end{equation}

The new network topology whose $Perf_Green$ value equal the min max one, we say that this new topology is the ``most robust'' for
all the possible traffic matrix.

As well, we will take an example to explain how to choose the link to close in our experiment. Figure shown as following, Three 
hostes: A, B, C, Three links repectively capacity of 3M, 4M and 2M. For simpleness, we suppose there are two traffic matrix: one 
trace 2M from A to B, and 1M from A to C; the other trace 1M from A to B, and 1M from A to C. For each traffic matrix, the 
optimal route is obvious, we will trace 2M from A to B across the lower link and trace the 1M from A to C across the upper one 
for the first traffic matrix, whose link utilization is 0.5. we will trace all the traffic acorss the lower link for the second 
traffic matrix, whose link utilization is 0.5 as well.

Now for some reason, we will choose one link to shut down for power saving without lose connection of the network. So two choice 
put before us, cut the upper link or the lower link. Let us take a little caluclation: when close the upper one, we should 
trace all the traffic acorss the lower link, as a result, in the first traffic matrix the link utilization is 0.75 and the other
is 0.5; when close the lower link, we should trace all the traffic acorss the upper link, in the first traffic matrix the link 
utilization is 1 and the other is 0.667. So according to our theory, the $PerfGreen(T)$ should be 1.5 and the new network 
topology will be the one which 3M link is closed.



\section{Conclusion}
The conclusion goes here.

\begin{thebibliography}{1}

\bibitem{IEEEhowto:kopka}
H.~Kopka and P.~W. Daly, \emph{A Guide to \LaTeX}, 3rd~ed.\hskip 1em plus
  0.5em minus 0.4em\relax Harlow, England: Addison-Wesley, 1999.

\end{thebibliography}

% that's all folks
\end{document}


